% !TEX TS-program = xelatex
%
\documentclass{article}
\usepackage[T1]{fontenc}

\usepackage[utf8]{inputenc}
\usepackage[english]{babel}
\usepackage{amsmath}
\usepackage{upgreek}
\usepackage{enumerate}

\begin{document}
	\title{Exercise Sheet 5 - Documentation}
	\date{}
	\author{Dötlinger Lukas, Kaltschmid Michael, Reiter Markus}

	\maketitle

  \section{Thread Safety}
  
  \section{Sharing Objects}
  
    \subsection{Publication and Escape}
    
    This example shows how the \texttt{this} can escape during construction and how to avoid this.\\
    The class \texttt{ThisEscape.java} lets the this escape during construction since it registers on construction a \texttt{EventListener}, where the \texttt{onEvent()} method is overwritten by the \texttt{doSomething()} method of the \texttt{ThisEscape} object, in the \texttt{EventSource} Thread, which enables the thread to call methods on the \texttt{ThisEscape} object (the "this" that escaped), although it is not fully constructed. This is problematic because thread behaviour is not fixed!\\
    The \texttt{EventSource} is a thread that always waits for new listeners and immediately calls their \texttt{onEvent()} method, which executes the \texttt{doSomething()} method of the underlying \texttt{ThisEscape} object, since it was overwritten on construction.\\ 
    If the \texttt{ThisEscape.java} is executed, it will print out the times when a race condition occours, which is always, since the \texttt{num} variable of the \texttt{ThisEscape} object is accessed by the \texttt{EventSource} thread before it is initialized in the constructor.\\
    \\
    The \texttt{SafeListener.java} avoids race conditions, since new instances of the class are created by a static method that adds them to the \texttt{EventSource} thread after the constructor is called. Therefore the object is correctly constructed before it is used. So this does not escape!\\

  
\end{document}
