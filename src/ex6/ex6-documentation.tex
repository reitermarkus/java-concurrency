% !TEX TS-program = xelatex
%
\documentclass{article}
\usepackage[T1]{fontenc}

\usepackage[utf8]{inputenc}
\usepackage[english]{babel}
\usepackage{amsmath}
\usepackage{upgreek}
\usepackage{enumerate}
\usepackage[binary-units=true]{siunitx}
\usepackage{setspace}% http://ctan.org/pkg/setspace
\usepackage{ragged2e}

\title{Exercise Sheet 6 - Documentation}
\date{}
\author{Dötlinger Lukas, Kaltschmid Michael, Reiter Markus}

\begin{document}
  \RaggedRight

  \maketitle

  \section{Composing Objects}
    The task in this excercise was to protect the $lower < upper$ invariant. The given class used \texttt{AtomicInteger} for these two values. This is not sufficient, however, since you need to synchronize on both values to also keep the invariant intact. We need to synchronize both because it is possible for two threads to break the invariant. For example, if one thread reads \texttt{upper}, then another thread writes \texttt{upper}, but the first thread only now sets the new value of \texttt{lower} based on the now outdated value of \texttt{upper}. Simply using an \texttt{AtomicInteger} is therefore not enough.

    To implement the generation of random ranges, we used the \texttt{ThreadLocalRandom} class with its \texttt{nextInt} method. We create 4 threads, each of which loops forever. In this loop, a random \texttt{lower} and \texttt{upper} are generated, and then set using the appropriate methods of the \texttt{NumberRange} class. Since with random numbers there is a high likelyhood that $lower < upper$ does not hold, we simply catch the \texttt{IllegalArgumentException} and retry in each iteration. Then there is another thread which generates a random number and tests if it is in the random range using the \texttt{isInRange} method.

  \section{Scalability}
    The task was to measure execution time of the provided program with different directories and furthermore write a concurrent version of the program.
    \\[3pt]
    Benchmarking turns out be rather difficult because there are many variables to consider. For example on any first run of the program you are likely to be IO bound rather then CPU bound which more or less negates the benefit of parallelism. The type of storage used is therefore a huge factor. HDDs skew the results more than SSDs in this regard.
    \\[3pt]
    So in order to get somewhat accurate results we decided to run the sequantial and parallel version of the program twice to ensure traverses beeing cached and therefore only focus on CPU results.
    \\[3pt]
    The results were mixed. We could measure speedups of up to 4 when running in parallel with 12 threads. However the average speedups was a lot slower. The reason for this lies in the implementation and data to traverse. The approach for the parallel programm was to partition all folders of the root in equally sized arrays of paths and subsequently compute the size of every file recursively in those arrays on every thread.
    \\[3pt]
    One can already guess that this approach has its flaws because depending on the folder structure some threads have more work than others. So in a perfect world where all root folders have the same folder structure with the same files a speedup of n where n is the amount of logical cores, could be possible.
    \\[24pt]
    \begin{doublespacing}
      \begin{align*}
        \begin{tabular}{|*{3}{c|}}
          \hline
          \multicolumn{1}{|c}{\bfseries Sequential:} & \multicolumn{1}{c}{} & \multicolumn{1}{c|}{} \\
          \hline
          $folder$ & $size$ & $time$ \\
          \hline
          Program Files (x86) & $\SI{642882.17}{\mega\byte}$ & $\SI{15740}{\milli\second}$ \\
          \hline
          Pictures & $\SI{982.78}{\mega\byte}$ & $\SI{23}{\milli\second}$ \\
          \hline
          Windows & $\SI{16615.42}{\mega\byte}$ & $\SI{5572}{\milli\second}$ \\
          \hline
        \end{tabular}
      \end{align*}
    \end{doublespacing}
    \begin{doublespacing}
      \begin{align*}
        \begin{tabular}{|*{3}{c|}}
          \hline
          \multicolumn{1}{|c}{\bfseries Parallel:} & \multicolumn{1}{c}{} & \multicolumn{1}{c|}{} \\
          \hline
          $folder$ & $size$ & $time$ \\
          \hline
          Program Files (x86) & $\SI{642882.17}{\mega\byte}$ & $\SI{5747}{\milli\second}$ \\
          \hline
          Pictures & $\SI{982.78}{\mega\byte}$ & $\SI{5}{\milli\second}$ \\
          \hline
          Windows & $\SI{16615.42}{\mega\byte}$ & $\SI{3649}{\milli\second}$ \\
          \hline
        \end{tabular}
      \end{align*}
    \end{doublespacing}
\end{document}
