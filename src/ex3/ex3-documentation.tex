% !TEX TS-program = xelatex
%
\documentclass{article}
\usepackage[T1]{fontenc}

\usepackage[utf8]{inputenc}
\usepackage[english]{babel}
\usepackage{amsmath}
\usepackage{upgreek}
\usepackage{enumerate}
\usepackage{setspace}% http://ctan.org/pkg/setspace

\begin{document}
	\title{Exercise Sheet 3 - Documentation}
	\date{}
	\author{Dötlinger Lukas, Kaltschmid Michael, Reiter Markus}

	\maketitle

  \section{Monitor Locks}

    The \texttt{Database} class implemented using a \texttt{ReentrantReadWriteLock}, which allows either a single writer or multiple readers to keep the lock. A \texttt{ReentrantReadWriteLock} consists of two separate locks, a read lock and a write lock.
    The \texttt{Test} class creates some writer and reader threads based on a given matricular number. A \texttt{Database} called \texttt{db} is created, as well as fixed size integer array.
    A writer thread calls \texttt{db.acquireWrite()}, writes a random integer to a random position in the array and then calls \texttt{db.releaseWrite()}.
    A reader thread calls \texttt{db.acquireRead()}, reads from a random position in the array and then calls \texttt{db.releaseRead()}.
    A writer thread increases the variable \texttt{rwcur} and \texttt{wwcur}, a reader thread increments \texttt{wrcur}.
    A writer thread resets \texttt{wrcur} and \texttt{wwcur} and calculates the average and maximum of this value, a reader thread does the same with \texttt{rwcur}.
    All of these actions are executed in a \texttt{while(!shutdown)} loop, and after a specified timeout in the main thread, \texttt{shutdown} is set to \texttt{true}, which causes the loops to break. The main thread then waits for all reader/writer threads to finish and prints the calculated maximum/average values.

  \section{Wait/Notify}

  \section{Semaphore/Locks}

    The java class \texttt{Semaphore.java} is a custom semaphore implementation using \texttt{Locks} in java. When initializing an instance of the semaphore, the constructor sets the capacity to the sum of digits of a randomly chosen matricular number of one of the authors. The capacity represents the number of available resources. The \texttt{Semaphore} class also has a \texttt{Lock} lock and a \texttt{Condition} condition for that lock.\\
    In the \texttt{main}-function, we initialize one instance of the \texttt{Semaphore} class and create a few threads, which are all executing the following operations: Acquire a random number of resources, print the status and then release the acquired resources.\\
    Acquiring a resource is done by calling the function \texttt{void P(int x)}. It is calling \texttt{lock.lock()} and then checks if the number of resources (x) is available. If yes, it acquires the resources and releases the lock. If not, it calls \texttt{condition.await()}. Releasing a resource is done by calling the function \texttt{void V(x)}. It locks the lock, then releases the resources and calls \texttt{condition.signalAll()}, to notify all waiting threads, before it unlocks the lock again.

  \section{Synchronized/Atomic}

    The \texttt{UpdateAtomic} is implemented using an \texttt{AtomicInteger} variable. In \texttt{public int modify(int val)}, the counter is incremented using \texttt{counter.addAndGet(val)} and returned. In the \texttt{UpdateSync} class, \texttt{synchronized} is used to make the \texttt{counter += val} operation in \texttt{synchronized int modify(int val)} atomic. Both classes are tested by running 10 threads which are adding random numbers between 1 and 6 using \texttt{modify} and then printing the incremented value afterwards.

\end{document}
